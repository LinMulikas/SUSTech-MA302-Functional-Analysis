\documentclass{article}

\usepackage[utf8]{inputenc}
\usepackage{amsthm}
\usepackage{amsmath}
\usepackage{amssymb}
\usepackage{amsfonts}
\usepackage{graphicx}
\usepackage{tikz}
\usepackage{authblk}

% 习惯最小的题号用自定义的id命令直接打出来
% TODO: Self definitions.
% \newcommand{\id}{\subsubsection*}
% \newcommand{\fc}{\frac}

\usepackage{geometry}
% 纸张尺寸大小
% 也可以手动写 left, right, top, bottom 参数
\geometry{a4paper, scale = 0.8}
%\geometry{left = 2.54cm, right = 2.54cm, top = 3.17cm, bottom = 3.17cm}

\usepackage{indentfirst}
\renewcommand{\baselinestretch}{1.2}



%TODO: Title
\title{MA302 Homework 4}

%TODO: Author
\author{WANG Duolei, SID:12012727}
\affil{wangdl2020@mail.sustech.edu.cn}



%TODO: Date
\date{}

%TODO: Theorem, lemma


%TODO: Main document
\begin{document}

\maketitle

% 3.1
\subsection*{3.1 Solution}
\begin{itemize}
    \item Translation invariant.
    
    Considering
    \begin{align*}
        d(x + z, y + z) &= \|x + z - (y + z)\|\\
        &= \|x - y\|\\
        &= d(x, y)\\
    \end{align*}

    \item Homogeneous.
    
    Considering
    \begin{align*}
        d(tx, ty) &= \|tx - ty\|\\
        &= |t|\|x - y\|\\
        &= |t|d(x, y)\\
    \end{align*}
\end{itemize}


% 3.5
\subsection*{3.5 Solution}
\begin{enumerate}
    \item The inequality.
    
    Considering the left inequality, it's clearly as \(m^p\) is actually a part of the right side. 

    Considering that \(|x_j|^p \le m,\ \forall j\), the right side is clearly holds.

    \item The limitation.
    
    Considering
    \begin{align*}
        {(m^p)}^{1/p} &\le (\sum_{j = 1}^n |x_j|^p)^{1/p} \le {(nm^p)}^{1/p} \\
        \implies \|x\|_{l^\infty} &\le \|x\|_{l^p} \le n^{1/p}\|x\|_{l^\infty}\\
    \end{align*}
    And \(\lim_{p \to \infty} n^{1/p} = 1,\ \forall n \ge 1\). Thus 
    \(\|x\|_{l^\infty} = \lim_{p \to \infty}\|x\|_{l^p}\)
\end{enumerate}


% 3.6
\subsection*{3.6Solution}
Firstly, considering that in a finite dimension vector space, the property
\[\|\cdot\|_{l^\infty} < \|\cdot\|_{l^p} \le n^{1/p}\|\cdot\|_{l^\infty}\]

holds.

And for arbitrary vector space, \(\mathbf{x} \in {l}^1\) implies it can minus some remainder to be approximate by finite dimension vector with arbitrary precise. Note the vector as \(x\)

Thus
\[\|\mathbf{x}\|_{l^\infty} - \varepsilon \le \|x\|_{l^\infty} \le \|x\|_{l^p} \le \|x\|_{l^\infty} + \varepsilon \le \|\mathbf x\|_{l^\infty} \]

and 
\[\|\mathbf{x}\|_{l^p} - \varepsilon \le \|x\|_{l^p} \le \|x\|_{l^\infty} + \varepsilon \le \|\mathbf x\|_{l^\infty} + \varepsilon\]

Considering the conclusion of 3.5

\[\|x\|_{l^\infty} \le \|x\|_{l^p}\]

which implies \(\|\mathbf x\|_{l^\infty} - \varepsilon \le \|\mathbf x\|_{l^p}\). Thus 
\[\|\mathbf x\|_{l^\infty} - \varepsilon \le \|\mathbf x\|_{l^p} \le \|\mathbf x\|_{l^\infty} + 2\varepsilon\]

Thus the limitation holds.

% 3.7
\subsection*{3.7 Solution}
Notice that \(p/q > 1\). And the series

\[\sum_{i = 1}^\infty \frac{1}{[n(\ln(n))^2]^k}\]

is converge for \(k \ge 1\) and diverges for \(k < 1\). Thus construct \(x_n = a_n\), where \(a_n\) is just the term of the series above.


% 3.8
\subsection*{3.8 Solution}
As \(U \subseteq X\) is a open normed vector space. There exist \(\varepsilon > 0,\ s.t.\ B(0, \varepsilon) \subseteq U\). Considering the vector space is closed onto the add, scalar-multiply, thus \(k B(0, \varepsilon) \subset U\). Considering
\[\forall x \in X,\ x \in \frac{2\|x\|}{\varepsilon}B(0, \varepsilon) \cap X\]
Thus \(x \in U\), which implies \(X \subset U\). Thus \(U = X\).


% 3.9
\subsection*{3.9 Solution}
Firstly, \(\bar U\) is closed. We need to check whether it's linear subspace. Considering two sequences converges in \(\bar U\), \((x_n) \to x,\ (y_n) \to y\). Then \((x_n + y_n) \to x + y,\ (k x_n) \to kx\) is clearly as the sequence space of \(\bar U\) is subspace of \(\bar U\), thus it's a linear subspace. 




% 3.11
\subsection*{3.11 Solution}
Firstly,
\[\|f_n - f\|_{L^p}^p = \int_I |f_n(x) - f(x)|^p dx \le \int_I \|f_n - f\|_{\infty}^p dx = \|f_n - f\|_\infty^p\]

thus, \(\|f_n - f\|_{\infty} \to 0 \implies \|f_n - f\|_{\infty}^p \to 0 \implies \|f_n - f\|_{L^p}^p \to 0 \implies \|f_n - f\|_{L^p} \to 0\).

Secondly,
\[|f_n(x) - f(x)| \le \|f_n - f\|_{\infty}\]

thus it's clearly that \(\|f_n - f\|_{\infty} \to 0 \implies |f_n(x) - f(x)| \to 0\).



% 3.12
\subsection*{3.12 Solution}
To prove \(c_0(\mathbb K)\) is separable, we can prove it has a countable set with a dense linear span. Here I claim that the set
\[E := \bigcup_{k = 1}^\infty E_k,\ E_k^i := (0, \ldots, x^i, \ldots)\]

where the k-th coordinate has nonzero the i-th term of sequence \(\{\frac{1}{2^i}\}_{i = 1}^\infty\) satisfies the requirement.

It can be checked as every null sequence can be approximate in the order of coordinates with the help of binary division. Thus \(c_0(\mathbb K)\) is separable.

% 3.13
\subsection*{3.13 Solution}
\begin{itemize}
    \item Sufficient.
    
    Considering \(X\) is separable, then for every vector \(y \in Y\), as \(T\) is bijection, \(T^{-1}(y) \in X\). Note the countable dense subset as \(E = \{x_i\}_{x = 1}^\infty\). Consider \(T(E)\) is countable, we need to check whether it's dense.
    
    There exist a sequence \((x_i)_{i = 1}^{\infty} \subset E\) such that
    
    \[\|x_i - T^{-1}(y)\|_X \to 0\]
    \
    Then,
    \[c_1 \|x_i - T^{-1}(y)\|_X \le \|T(x_i) - y\|_Y \le c_2 \|x_i - T^{-1}(y)\|_X \implies \|T(x_i) - y\|_Y = 0\]
    \
    Thus the set \(T(E) \subset Y\) is the countable dense subset.

    \item Necessary.
    
    As the same, and use the truth
    \[c_1' \|y\|_Y \le \|T^{-1}(y)\|_X \le c_2' \|y\|_Y \]
    one can prove it as same as the sufficient.
\end{itemize}




% 3.16
\subsection*{3.16 Solution}

\begin{itemize}
    \item Sufficient.
    
    Consider \((X, \|\cdot\|)\) is separable, then it has a countable dense subset, note it as \(E = \{x_1, x_2, \ldots\}\). Considering 
    \[X_i := \mathrm{Span}(x_i)\]
    Then \(X = \mathrm{Span}(E) = \bigcup_{i = 1}^\infty X_i\), which is just the hypothesis.

    \item Necessary.
    
    Considering \(X\) can be written as the form. Notice that every \(X_i\) is finite dimension, we can find a basis, note it as \(E_i := \{v_i^1, v_i^2, \ldots, v_i^{n_i}\}\). 
    
    Then
    \[X = \bigcup_{i = 1}^\infty X_i = \bigcup_{i = 1}^\infty \mathrm{Span}(E_i)\]

    And considering \(E := \bigcup_{i = 1}^\infty E_i\) must be contable, thus

    \[X = \bigcup_{i = 1}^\infty \mathrm{Span}(E_i) = \mathrm{Span}(E)\]

    the set \(E\) is just a countable dense subset.
\end{itemize}

\end{document}