\documentclass{article}

\usepackage[utf8]{inputenc}
\usepackage{amsthm}
\usepackage{amsmath}
\usepackage{amssymb}
\usepackage{amsfonts}
\usepackage{graphicx}
\usepackage{tikz}
\usepackage{authblk}

% 习惯最小的题号用自定义的id命令直接打出来
% TODO: Self definitions.
% \newcommand{\id}{\subsubsection*}
% \newcommand{\fc}{\frac}

\usepackage{geometry}
% 纸张尺寸大小
% 也可以手动写 left, right, top, bottom 参数
\geometry{a4paper, scale = 0.9}
%\geometry{left = 2.54cm, right = 2.54cm, top = 3.17cm, bottom = 3.17cm}

\usepackage{indentfirst}
\renewcommand{\baselinestretch}{1.2}



%TODO: Title
\title{MA327 Homework 1}

%TODO: Author
\author{WANG Duolei, SID:12012727}
\affil{wangdl2020@mail.sustech.edu.cn}



%TODO: Date
\date{}

%TODO: Theorem, lemma


%TODO: Main document
\begin{document}
\maketitle
\begin{enumerate}
    \item[2.2] 
    Considering the following two cases
    \begin{enumerate}
        \item The case \(p = \infty\).
        
            \begin{itemize}
                \item Positive definite. 
                
                It's clearly because that every metric in \(X_i\) itself is positive define, and there are only absolute value, power function used in all the definition 
    
                \item Symmetric. Trivial.
        
                \item Traingle Inequality.
                
                Considering that
                \begin{align*}
                    &\max_i d_i(x_i, y_i)\\
                    \le &\max_i \{d_i(x_i, z_i) + d_i(z_i, y_i)\}\\
                    \le &\max_i \{d_i(x_i, z_i)\}
                \end{align*}
                
    
            \end{itemize}
    
        \item The case \(1 \le p < \infty\).
        
            \begin{itemize}
                \item Positive definite. It's clearly.
                \item Symmetric is trivial.
                \item Traingle Inequality.
                
                Considering that
                \begin{align*}
                    d(x, y) &= {[{d_1(x_1, y_1)}^p + {d_2(x_2, y_2)}^p + \cdots + {d_n(x_n, y_n)}^p]}^{1/p}\\
                    &\le {[{(d_1(x_1, z_1) + d_1(y_1, z_1))}^p + {(d_2(x_2, z_2) + d_2(y_2, z_2))}^{1/p }\cdots + {(d_n(x_n, z_n) + d_n(y_n, z_n))}^p]}^{1/p}\\
                    &\le {[d_1(x_1, z_1) + d_2(x_2, z_2) + \cdots + d_n(x_n, z_n)]}^{1/p} +{[d_1(x_1, y_1) + d_2(x_2, y_2) + \cdots + d_n(x_n, y_n)]}^{1/p} \\
                    &= d(x, z) + d(y, z)
                \end{align*}
                The last inequality can be proved by the hint with the htlp of mathematical induction.
            \end{itemize}
    

    \end{enumerate}


\item[2.3]
    \begin{itemize}
        \item Positive definite.
        Considering that
        \[\hat d = 1 - \frac{1}{1 + d} \ge 0\]
        and \(\hat d = 0 \iff d = 0 \iff x = y\).

        \item Symmetric. Trivial.
        \item Traingle Inequality.
        
        Let \(a = d(x, z), b = d(y, z), c = d(x, y), f(x) = x/(1 + x)\), it's clearly that \(a, b, c \ge 0\). And let \(f(a) = 1/(1 + d(x, z)), b = 1/(1 + d(y, z)), c = 1/(1 + d(x, y))\), we have already know that \(a + b \ge c\). Assume \(a + b = k \ge c\), firstly, considering the minimum of \(f(a) + f(b)\), note as \(h(a)\), we have
        \[h(a) = 1 - \frac{1}{a + 1} + 1 - \frac{1}{(1 + k) - a}\]
        Considering the derivation,
        \[h'(a) = (k + 2)\frac{(-2a + k)}{{(a + 1)}^2{[a - (k + 1)]}^2}\]
        thus, 
        \[f(a) + f(b) = h(a) \ge \min_a h(a) = h(k/2) = \frac{k}{1 + k/2} \ge \frac{k}{1 + k} = f(c)\]
        Thus, the traingle inequality holds.
    \end{itemize}


\item[2.4]
    \begin{enumerate}
        \item[(1).] The \(d\) is a metric for \(s(\mathbb K)\).
        
        Firstly, we have proved in 2.3 that \(\hat d = d_0/(1 + d_0)\) is a metric for every \(d_0\) as metric. Thus, we need to prove the traingle inequality for \(d_0(x, y) = |x - y|,\ x, y \in \mathbb{R}\).

        \[d(x, y) = \sum_{i = 1}^{\infty} \hat 2^{-i} \hat d(x_i, y_i) \le \sum_{i = 1}^{\infty} 2^{-i} {(\hat d(x, z) + \hat d(y, z))} = d(x, z) + d(y, z)\]

        Thus, the traingle inequality holds.

        \item[(2).] In a metric space, \({\{x\}}_{i = 1}^\infty \to y \iff {\{x_i\}}_{i = 1}^\infty \to y_i\).
            \begin{itemize}
                \item The forward is clearly, that if there exist a coordinate \(x_i\) doesn't converges to \(y_i\).
                    \[d(x, y) \ge d(x_i, y_i)\]
                    thus, \(d(x, y)\) can't converges to 0.
                \item The backward is according to 
                    \[d(x, y) \le \sum_{i = i}^{N} d(x_i, y_i)\]
                    Therefore, for every \(\varepsilon > 0\), considering there exist corresponding \({\{N_i\}}_i^N\) such that \(d(x_i, y_i) < \varepsilon/N\). Let \(N_0 = \max\{N_1, \dots, N_N\}\), thus
                    \[\forall\ n \ge N_0, d(x, y) \le \sum_{i = 1}^{N} d(x_i, y_i) \le N \varepsilon/N = \varepsilon\]
                    which means that \({\{x_i\}}_{i = 1}^\infty \to y\).
            \end{itemize}
    \end{enumerate}



\item[2.5]
    \begin{enumerate}
        \item[(1)] Finite union \(\bigcup_{i = 1}^N F_i\) is closed i.f.f \(\bigcap_{i = 1}^N F_i^c\) is open.
            Considering an arbitrary point in the finite intersection of complement, as every complement set is open, there exist \(\{B(x, r_1), \ldots, B(x, r_N)\}\) such that \(B(x, r_i) \subset F_i^c \). Considering the intersection of \(B(x, r_i)\) is also open. Thus 
            \[B(x, r_0) \subset \bigcap_{i = 1}^N F_i^c,\ r_0 := \min\{r_1, \ldots, r_N\}\]
            Thus, \(x\) is also a interior. Thus, it's open, which also means the finite union of closed set is closed.
            
        \item[(2)] Finite intersection. Also considering the complement \(\bigcup_{i = 1}^N F_i^c\), it's clearly that the countable infinite union of open set is union, thus the finite must also, which also means the finite intersection is closed.
    \end{enumerate}

\end{enumerate}

\end{document}